\documentclass[11pt]{beamer}
% \documentclass[11pt,handout]{beamer}
\usepackage[T1]{fontenc}
\usepackage[utf8]{inputenc}
\usepackage{float, afterpage, rotating, graphicx}
\usepackage{epstopdf}
\usepackage{longtable, booktabs, tabularx}
\usepackage{fancyvrb, moreverb, relsize}
\usepackage{eurosym, calc}
\usepackage{amsmath, amssymb, amsfonts, amsthm, bm} 


\usepackage[
    natbib=true,
    bibencoding=inputenc,
    bibstyle=authoryear-ibid,
    citestyle=authoryear-comp,
    maxcitenames=3,
    maxbibnames=10,
    useprefix=false,
    sortcites=true,
    backend=biber
]{biblatex}
\AtBeginDocument{\toggletrue{blx@useprefix}}
\AtBeginBibliography{\togglefalse{blx@useprefix}}
\setlength{\bibitemsep}{1.5ex}
\addbibresource{refs.bib}



\hypersetup{colorlinks=true, linkcolor=black, anchorcolor=black, citecolor=black, filecolor=black, menucolor=black, runcolor=black, urlcolor=black}

\setbeamertemplate{footline}[frame number]
\setbeamertemplate{navigation symbols}{}
\setbeamertemplate{frametitle}{\centering\vspace{1ex}\insertframetitle\par}


\begin{document}

\title{Skills and Wages}

\author[Poooja Bansal]
{
{\bf Poooja Bansal}\\
{\small University of Bonn}\\[1ex]
}


\begin{frame}
    \titlepage
    \note{~}
\end{frame}


\begin{frame}[t]
    \frametitle{Introduction}
    \begin{itemize}
     Many empirical studies have recognized the importance of
non-cognitive skills along with the cognitive skills and we build on this
by :

\item first examining whether cognitive and non-cognitive skills explain
difference in hourly wages after controlling for experience and
schooling

\item Secondly we analyze how these skills affect the wage profiles of
individuals across different occupational levels, since the
requirements for these skills vary with different occupations.
     \end{itemize}
    \note{~}
\end{frame}

\begin{frame}[t]
  \begin{itemize}
People typically embody both type of skills: Cognitive skills driving their reasoning
and thinking; and non-cognitive skills incorporating their personality traits.

\item There are numerous studies that have established measurements for cognitive
abilities, for instance AFQT scores, GAT scores test and IQ performance tests by
DIW,

\item Many economists have produced large body of evidence that employers in labor
market have now recognized the relationship between non-cognitive skills and
productivity. This recognition have led to the evolution of measures like
Rosenberg Self esteem scale and Rotter Locus of control, the Big Five Factor
Model.
 \end{itemize}
\end{frame}

\begin{frame}[t]
  \begin{itemize}
\item But to what extent is occupation useful to understand how education
and skills are related with wages?

\item With the rapidly changing trends in the global labor market,
employers today want their employees to possess a certain degree of
qualification, in terms of skills, education and experience, due to the
non-pecuniary characteristics of different jobs.

\item And in this highly competitive market, employees are keen to develop
their qualifications to suit the market needs. Hence we see how these
qualifications change in the occupational hierarchy.
 \end{itemize}
\end{frame}

\begin{frame}[t]
    \frametitle{Literature}
    \begin{itemize}
    \item There have been numerous studies which investigated the effect of
cognitive skills and personality traits on wages.
   \item \citet{heineck} confirms that employers highly value
individuals’ skills. \citet{farkas} also confirm that
employers assess cognitive and non-cognitive skills for hiring,
promotion and wage setting policies.
      
     \end{itemize}
    \note{~}
\end{frame}


\begin{frame}[t]
    \begin{itemize}
      On one hand, some studies suggested substantial returns to
cognitive skills :
   \begin{itemize}
\item Anger & Guido Heineck (2005) have established a positive
relationship between cognitive skills and labor market outcomes,
suggesting that abilities are correlated to the wages in a significantly
positive way for German workers.

\item Murnane, Willett & Levy (1995) also recognized the importance of
cognitive skills in wage determination.
\end{itemize}
While on the other hand, many research works found that cognitive
ability has a very little or no effect on earnings:
\begin{itemize}
Cawley, Heck- man & Vytlacil (2001) and Zax & Rees (2002) reported
that the effect of cognitive skills is much smaller than what has been
asserted by previous analyses and is rather a poor estimator of
earnings.
     \end{itemize}
    \note{~}
\end{frame}


\begin{frame}[t]
    \begin{itemize}
     \item Gintis & Osborne( 2001) explained how some personality traits
matter for employers because they facilitate the production of effort
at work and affect labour productivity.

\item Heckman,Stixrud & Urzua (2006) suggested that non- cognitive skill is
an equally strong determinant, if not more, as cognitive skill.

\item Bowles & Gintis (1976) and Edwards (1976) in their work showed
that skills such as dependability and persistence are highly valued by
employers.
     \end{itemize}
    \note{~}
\end{frame}


\begin{frame}[t]
 \frametitle{Expectations}
 \begin{itemize}
 \item We expect cognitive skills to have either positive or no association
with the earnings.

\item With respect to the personality traits used, we expect no significant
relation between Extraversion and wages, a positive relationship for
Openness and conscientiousness and negative for Neuroticism and
Agreeableness.

\item For occupations, we expect cognitive skills to be either positively or
not related to the occupational categories. For personality traits, we
expect more heterogeneous results for each category of occupation,
depending on their work task and roles.
     \end{itemize}
    \note{~}
\end{frame}

\begin{frame}[t]
    \begin{itemize}
 \frametitle{Literature}
     \end{itemize}
    \note{~}
\end{frame}


\begin{frame}[t]
    \begin{itemize}
 \frametitle{Literature}
     \end{itemize}
    \note{~}
\end{frame}


\begin{frame}[t]
    \begin{itemize}
 \frametitle{Literature}
     \end{itemize}
    \note{~}
\end{frame}

\begin{frame}[t]
    \begin{itemize}
 \frametitle{Literature}
     \end{itemize}
    \note{~}
\end{frame}

\begin{frame}[t]
    \begin{itemize}
 \frametitle{Literature}
     \end{itemize}
    \note{~}
\end{frame}



% Print black screen only in presentation mode for finishing up.
\mode<beamer> {
    \beamersetaveragebackground{black}
    \begin{frame}
        \frametitle{}
    \end{frame}

    \beamersetaveragebackground{white}
}

\begin{frame}[allowframebreaks]
    \frametitle{References}
    
    \renewcommand{\bibfont}{\normalfont\footnotesize}
    \printbibliography
    
    
\end{frame}

\end{document}
